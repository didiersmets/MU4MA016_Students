\documentclass[a4paper,12pt]{article}

%%%%%%%%%%%%%%%%%% Extensions %%%%%%%%%%%%%%%%%%%

\usepackage{a4wide}
\usepackage{graphicx}
\usepackage{float}
\usepackage{amssymb}
\usepackage{amsmath}
\usepackage{amsthm}
\usepackage{color}
\usepackage{array}
\usepackage{pstricks}
\usepackage{eucal}
\usepackage{mathrsfs}
%\usepackage[T1]{fontenc}
\usepackage[utf8]{inputenc}
\usepackage{layout}
\usepackage{fancyhdr}


%%%%%%%%%%%%%%%%%% Notations %%%%%%%%%%%%%%%%%%%

\newcommand{\nc}{\newcommand}
\nc{\dsp}{\displaystyle}
\nc{\pprime}{\prime\prime}
\nc{\mrm}{\mathrm}
\nc{\lbr}{\lbrack}
\nc{\rbr}{\rbrack}
\nc{\R}{\mathbb{R}}
\nc{\Z}{\mathbb{Z}}
\nc{\bfx}{\mathbf{x}}
\nc{\bfs}{\mathbf{s}}
\nc{\bfn}{\mathbf{n}}
\nc{\bfe}{\mathbf{e}}
\nc{\ora}{\overrightarrow}
\nc{\Dx}{\Delta x}
\nc{\Dt}{\Delta t}
\renewcommand{\headrulewidth}{0pt}

%%%%%%%%%%%%%%%%%%%%%%%%%%%%%%%%%%%%%%%%%%%%%%%%%

\title{Tutorial class 1}
\date{}

\begin{document}


\quad\\[10pt]

\centerline{\huge TP 0: shell + hello world}

\centerline{\rule{5cm}{1pt}}

\thispagestyle{fancy}
\fancyhead[L]{
  Master Sciences Technologie Santé\\ 
  Mention Mathématiques et Applications
}

\fancyhead[R]{
  MU4MA016\\
  2024-2025}


\quad\\[-5pt]
\section{Discovering the shell}



\noindent 
The goal of this exercise is to get familiar with the so-called ``shell'' running in a command window. All of the steps below except for the first one could/should be done without using the mouse (if you wish to fully apply to this rule it implies not  using a browser to search for help on the internet...).  Help within the shell can be obtained using the {\tt man} command (a shorthand for {\it manual}), and the shell itself is called {\it bash} (shorthand for {\it Bourne again shell}), so {\tt man bash} will give you access to the (huge) description of it. Help on individual commands can be obtained by {\tt man} followed by the name of the command.\\

 
\begin{itemize}

\item[1)] Open a terminal window.

\item[2)] Read the manual of the commands \texttt{cd} and \texttt{pwd}. 

\item[3)] Navigate to the root directory \texttt{/}, read the manual of the command \texttt{ls}, and then list the content of the root directory (all entries and with long listing format). Are these listed entries files or directories ? Who is the owner of these and would you be able to read and/or write over them ?

\item[4)] Navigate back to you home directory, list all of its entries using {\tt ls}, and place them into a file called {\tt file1.txt}. For that purpose, explore the possibilities of redirecting the shell output(s) into files (inside the bash manual search for the section called REDIRECTION, searches like in the Vim editor are done using the slash key {\tt /} followed by the searched string).

\item[5)] Read the manual for the command {\tt mkdir}. From your home directory, create a subdirectory named {\tt MA016}, and inside the latter another one named {\tt tp0} (can you do both at once ?).

\item[6)] Read the manual for the commands {\tt cp} and {\tt mv}. Make a copy of the file {\tt file1.txt} inside {\tt MA016/tp0}, which you will name {\tt file2.txt}. Then move {\tt file1.txt} into {\tt MA016/tp0}. Navigate into {\tt MA016/tp0} and check that both files inside it are identical (using the {\tt diff} command).   

\item[7)] Edit {\tt file2.txt} using an editor ({\tt vim} and {\tt emacs} are top choices), apply some changes, and then observe how the difference is reported by {\tt diff}.

\item[8)] Erase both of these files now, so that we have a clean directory for real work. 

\end{itemize}

\noindent Other frequently used shell commands include : {\tt pwd}, {\tt cat}, {\tt more}, {\tt ps -aux}, {\tt top}, {\tt grep} or {\tt egrep}. The use of pipes {\tt cmd1 | cmd2 } that bridge the output of {\tt cmd1} to the input of {\tt cmd2} is also a powerful feature to be aware of. 

\medskip\noindent
You will find tons of information online about the shell, feel free to now use a
browser and a mouse in your learning process (but remember about the existence
of the {\tt man} command, and the fact not all online resources are exact...).

\medskip


\section{Hello world in C}

\begin{itemize}

\item[1)] Create the source code file {\tt hello.c} for a program that will simply print "Hello, World!" to the screen.
\item[2)] Compile that program into an executable named {\tt hello}, using the
	GNU C compiler {\tt gcc} directly.
\item[3)] Compile it instead using the CMake tool (documentation online). This may seem tedious at
	first, but it will save you a lot if time when your projects will get
		bigger, with many source files and/or third party code.

\end{itemize}

\quad\\
This is just the starting point of your journey with C/C++. There are also tons of information online, and some good printed books to (ask me depending on your present knowledge of these langages). For a quick reference, {\tt cppreference.com} and {\tt cplusplus.com} are recommended online sources.\\


\section{Pascal's triangle and console output}
\noindent
Write a C program that takes as command line argument an integer $n$ and then prints the first $n$ lines of Pascal's triangle in the terminal:
$$
\begin{array}{llll}
A_{0,0}\\
A_{1,0} & A_{1,1}\\
A_{2,0} & A_{2,1} & A_{2,2}\\
\vdots & \vdots & \vdots& \ddots
\end{array}
$$
using the recurrence formula $A_{i,j} = A_{i-1,j-1} + A_{i-1,j}$ (if $0 < j < i$, and $1$ otherwise). 


\section{Reading and writing into a file using C}

\begin{itemize}

\item[1)] Build a C program, with executable called \texttt{read\_file},
  which reads the content of a text file (whose name is a command line argument) 
  and print it line by line in the terminal. Test it over the file 
  \texttt{test-tp0.txt} included within this TP0.7z archive.\\
  
\item[2)] Let $n$ be a positive integer, $x_{\mrm{min}} = -6\pi$, $x_{\mrm{max}} = +6\pi$, and $
  \Delta x = (x_{\mrm{max}}-x_{\mrm{min}})/(n-1)$. For $j=0,\dots, n-1$, set  
  $x_j = x_{\mrm{min}} + j \Delta x$ and $y_j = \sin(x_j)/x_j$. Write a program that
  takes $n$ as input and writes a text file formatted as follows:\\[10pt]
  \begin{tabular}{lll}
    $x_0$ & &$y_0$\\
    $x_1$ & &$y_1$\\
    $\vdots$ && $\vdots$\\
    $x_{n-1}$ && $y_{n-1}$\\   
  \end{tabular}
  \quad\\[10pt]
  with a tabulation separating elements in the same line, and linebreaks to separate lines.\\

\item[3)] Make a graphical representation of the function $x\mapsto \sin(x)/x$ on the interval 
  $x\in \lbr-6\pi,+6\pi\rbr$ using GNU Gnuplot tracing program (see \texttt{http://www.gnuplot.info}).

\end{itemize}

\section{Using Git}
I have set up a Git repository for you to save your files during the whole
semester. That not only provides a backup, it will teach you how to deal with
version/history in larger projects, and also collaborative work.\\
The repository is hosted on Github, and you would traditionally clone it
using :
\begin{center}
{\tt git clone https://github.com/didiersmets/MU4MA016\_Students.git} 
\end{center}
That would only give you read access to it. I have therefore created a so-called token 
giving you (limited) write access until the end of January 2025. For that purpose, from 
your home directory in a terminal window clone the repository using 
the command :
\begin{center}
	{\tt git clone https://[token]@github.com/didiersmets/MU4MA016\_Students.git} 
\end{center}
where you replace 
\begin{center}
	{\tt [token] by }\\ 
	{\tt \footnotesize github\_pat\_11ATF6NQI0Hpq8GjJhLeql\_dKV8db1DpYSMeMs4nqljNkaAmlGVgPXEr92g8sGoKqmNP6Y6NAQOmVzrMEp}
\end{center}
That will create a directory called {\tt MU4MA016\_Students} inside your home
directory. Inside that directory, create a subdirectory whose name is your
family name (better avoid blank space in the name), in the sequel I call it
NAME. Organize your source files to your liking into that NAME directory, 
and then upload them to Github. For that purpose you will need to :
\begin{enumerate}
	\item
		Set your name and email into your local git config : from inside
		the {\tt MU4MA016\_Students} directory issue the command
		\begin{center}
		{\tt git config --local user.name "yournamewithoutspace"}
		\end{center}
		and then 
		\begin{center}
		{\tt git config --local user.email "youremail"}
		\end{center}
	\item Next, 
		\begin{center}
			{\tt git add NAME}
		\end{center}
		Check all files added for the next commit by
		\begin{center}
			{\tt git status}
		\end{center}
	\item Commit you changes with a commit message :
		\begin{center}
			{\tt git commit -m "The first commit of [yournamehere]"}
		\end{center}
	\item Finally push your commit to the central repository :
		\begin{center}
			{\tt git push}
		\end{center}
\end{enumerate}
Take a tour of a Git, either through the man command or using online tutorials. 
The goal is to get familiar with that fantastic tool. Note that {\tt git pull} 
will bring back the last centralized version of the repository, including your 
class mate files ! (on purpose, you can then learn by glancing at your class 
mate's code too).\\
{\bf You are requested to push to that NAME directory regularly (e.g. at least 
once a week) for uploading your work, for this and the later exercise sessions.} 
That is a safe practice, and also a way for me to supervise your work/progress.
\end{document}

